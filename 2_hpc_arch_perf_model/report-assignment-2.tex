\documentclass[a4paper,10pt]{article}
\usepackage[utf8]{inputenc}
\usepackage[a4paper,
            bindingoffset=0.2in,
            left=1in,
            right=1in,
            top=1in,
            bottom=1in,
            footskip=.25in]{geometry}

%###############################################################################

%\input{~/layout/global_layout}


%###############################################################################

% packages begin

\usepackage[
  backend=biber,
  sortcites=true,
  style=alphabetic,
  eprint=true,
  backref=true
]{biblatex}
\addbibresource{bibliographie.bib}
\usepackage[acronym]{glossaries}

\usepackage{euscript}[mathcal]
% e.g. \mathcal{A} for fancy letters in mathmode
\usepackage{amsmath,amssymb,amstext,amsthm}

\usepackage{mdframed}
\newmdtheoremenv[nobreak=true]{problem}{Problem}[subsection]
\newmdtheoremenv[nobreak=true]{claim}{Claim}[subsection]
\newtheorem{definition}{Definition}[subsection]
\newtheorem{lemma}{Lemma}[claim]
\newtheorem{plemma}{Lemma}[problem]

\usepackage{mathtools}
\DeclarePairedDelimiter\ceil{\lceil}{\rceil}
\DeclarePairedDelimiter\floor{\lfloor}{\rfloor}

\usepackage{enumerate}
\usepackage[pdftex]{graphicx}
\usepackage{subcaption}
% 'draft' für schnelleres rendern mitübergeben -> [pdftex, draft]
% dadruch wird nicht das bild mitgerendered, sondern nur ein kasten mit bildname -> schont ressourcen

\usepackage{hyperref}

\usepackage{tikz}
\usetikzlibrary{arrows,automata,matrix,positioning,shapes}

% for adding non-formatted text to include source-code
\usepackage{listings}
\lstset{language=Python,basicstyle=\footnotesize}
% z.B.:
% \lstinputlisting{source_filename.py}
% \lstinputlisting[lanugage=Python, firstline=37, lastline=45]{source_filename.py}
%
% oder
%
% \begin{lstlisting}[frame=single]
% CODE HERE
%\end{lstlisting}
\usepackage{algorithm}
\usepackage{algpseudocode}

\usepackage{wasysym}

\usepackage{titling}
\usepackage{titlesec}
\usepackage[nocheck]{fancyhdr}
\usepackage{lastpage}

\usepackage{kantlipsum}
\usepackage[colorinlistoftodos,prependcaption,textsize=tiny]{todonotes}

% packages end
%###############################################################################

\pretitle{% add some rules
  \begin{center}
    \LARGE\bfseries
} %, make the fonts bigger, make the title (only) bold
\posttitle{%
  \end{center}%
  %\vskip .75em plus .25em minus .25em% increase the vertical spacing a bit, make this particular glue stretchier
}
\predate{%
  \begin{center}
    \normalsize
}
\postdate{%
  \end{center}%
}

\titleformat*{\section}{\Large\bfseries}
\titleformat*{\subsection}{\large\bfseries}
\titleformat*{\subsubsection}{\normalsize\bfseries}

\titleformat*{\paragraph}{\Large\bfseries}
\titleformat*{\subparagraph}{\large\bfseries}

%###############################################################################

\pagestyle{fancy}
\fancyhf{}
% l=left, c=center, r=right; e=even_pagenumber, o=odd_pagenumber; h=header, f=footer
% example: [lh] -> left header, [lof,ref] -> fotter left when odd, right when even
%\fancyhf[lh]{}
%\fancyhf[ch]{}
%\fancyhf[rh]{}
%\fancyhf[lf]{}
\fancyhf[cf]{\footnotesize Page \thepage\ of \pageref*{LastPage}}
%\fancyhf[rf]{}
\renewcommand{\headrule}{} % removes horizontal header line

% Fotter options for first page

\fancypagestyle{firstpagestyle}{
  \renewcommand{\thedate}{\textmd{}} % removes horizontal header line
  \fancyhf{}
  \fancyhf[lh]{\ttfamily M.Sc. Computer Science\\KTH Royal Institute of Technology}
  \fancyhf[rh]{\ttfamily Period 4\\\today}
  \fancyfoot[C]{\footnotesize Page \thepage\ of \pageref*{LastPage}}
  \renewcommand{\headrule}{} % removes horizontal header line
}
%###############################################################################

\newcommand\extrafootertext[1]{%
    \bgroup
    \renewcommand\thefootnote{\fnsymbol{footnote}}%
    \renewcommand\thempfootnote{\fnsymbol{mpfootnote}}%
    \footnotetext[0]{#1}%
    \egroup
}

%###############################################################################

\title{
  \normalsize{DD2356 VT25 Methods in}\\
  \normalsize{High Performance Computing}\\
  \large{Assignment 2}
}
\author{
  \small Rishi Vijayvargiya\textsuperscript{\textdagger}\\[-0.75ex]
%  \footnotesize\texttt{MN: }\\[-1ex]
  \scriptsize\texttt{rishiv@kth.se}
  \and
  \small Paul Mayer\textsuperscript{\textdagger}\\[-0.75ex]
%  \footnotesize\texttt{MN: }\\[-1ex]
  \scriptsize\texttt{pmayer@kth.se}
  \and
  \small Lennard Herud \textsuperscript{\textdagger}\\[-0.75ex]
%  \footnotesize\texttt{MN: }\\[-1ex]
  \scriptsize\texttt{herud@kth.se}
}
\date{}

%###############################################################################
% define Commands

\newcommand{\N}{\mathbb{N}}
\newcommand{\R}{\mathbb{R}}
\newcommand{\Z}{\mathbb{Z}}
\newcommand{\I}{\mathbb{I}}

\newcommand{\E}{\mathbb{E}}
\newcommand{\Prob}{\mathbb{P}}

\renewcommand{\epsilon}{\varepsilon}

%###############################################################################
\makeatletter
\renewcommand*{\@fnsymbol}[1]{\ensuremath{\ifcase#1\or \dagger\or \ddagger\or
   \mathsection\or \mathparagraph\or \|\or **\or \dagger\dagger
   \or \ddagger\ddagger \else\@ctrerr\fi}}
\makeatother
%###############################################################################

\begin{document}
\maketitle
\extrafootertext{\textsuperscript{\textdagger}Authors made equal contribution to the project}
\thispagestyle{firstpagestyle}

\listoftodos
\vspace{1em}

% content begin
%

\section*{Prefix}
The code for our project can be found at this location: \url{https://github.com/paulmyr/DD2356-MethodsHPC/tree/master/2_hpc_arch_perf_model}. 

\tableofcontents
\newpage


\section{}
\todo[inline]{Check headers}



\section{Exercise 1: Supercomputer Architecture with hwloc}

\section{Exercise 2: Roofline Model}
\todo{How to build the roofline model? What should the array size in stream benchmark be? Using 200mil causes errors during compilation. The default size seems too small according to comments in the code}
\todo{How to establish the peak compute using the information of the processor?}

\section{Exercise 3: Modeling Sparse Matrix-Vector Multiply}

\section{Exercise 4: Measure the Performance with Perf}
\subsection{Matrix Multiply}
We created files \verb|matrix_multiply.c| and \verb|matrix_multiply_opt.c|, which were the default and optimized versions of the algorithm provided, respectively. We used the follogin \verb|salloc| command to allocate resources: 

\begin{lstlisting}[language=bash,basicstyle=\ttfamily]
salloc -t 00:30:00 --nodes 1  -A edu25.DD2356 -p shared
\end{lstlisting}
Since no \verb|--cpus-per-task| argument was provided, we believe that the default value of 1 was used. Additionally, since no \verb|-n| flag was provided either, we believe that the code was execute on 1 core. 

Compiling using the provided commands, then executed \verb|perf| using the following command: 
\begin{lstlisting}[language=bash,basicstyle=\tiny\ttfamily]
srun -n 1 perf stat -e instructions,cycles,L1-dcache-load-misses,L1-dcache-loads ./matrix_multiply.out
\end{lstlisting}
The same thing was done for the optimized version. We varied \verb|MSIZE| manually, recompiled, and then re-executed the commands above for the different \verb|MSIZE| values. 
Doing so, we obtained the following table (for \textit{Elapsed time}, we record the values obtained from \verb|perf| and not the one produced by the algorithm) 
\begin{table}[h!]
\centering
\begin{tabular}{|c|p{2.5cm}|p{2.5cm}|p{2.5cm}|p{2.5cm}|}
\hline
\textbf{Event Name} & \textbf{Naive (MSIZE 64)} & \textbf{Optimized (MSIZE 64)} & \textbf{Naive (MSIZE 1000)} & \textbf{Optimized (MSIZE 1000)} \\
\hline
\textbf{Elapsed time (sec)} & 0.008982713 & 0.008476052 & 17.968526050 & 1.685069089 \\
\hline
\textbf{Instructions per cycle} & 1.16 & 1.28 & 0.58 & 2.17 \\
\hline
\textbf{L1 cache miss ratio} & 0.1817 & 0.0610 & 0.5618 & 0.1649 \\
\hline
\textbf{L1 cache miss rate PTI} & 101.5736 & 26.5007 & 400.6424 & 131.9917  \\
\hline
\end{tabular}
\caption{Perf Stats for Default and Optimized Matrix-Multiply}
\end{table}
The raw outputs for the different values of \verb|MSIZE| for the \verb|matrix_multiply.out| and \verb|matrix_multiply_opt.out| binaries can be found in the file \verb|matrix_multiply_outputs.txt| in the repository \href{https://github.com/paulmyr/DD2356-MethodsHPC/blob/master/2_hpc_arch_perf_model/exercise4/matrix_multiply_outputs.txt}{here}. For the \verb|L1 cache miss ratio| and the \verb|L1 cache miss rate PTI|, we used the formulas \href{https://canvas.kth.se/courses/53216/pages/tutorial-the-perf-tool?module_item_id=1067474}{here}.

\todo{Answer the question for this section}

\subsection{Matrix Transpose}


\section{Bonus: Network Latency \& Bandwidth Analysis with MPI Ping-Pong}

\todo{The program hangs and output needs to be flushed out (despite recommendation on discussion board)}
\todo{The output is a bit flat for the provided sizes. Increasing ths size (to about 16384) causes segmentation fault at different points in time}
\todo{Submit with these annotations or try and "fix" the problem}

% content end
%###############################################################################

% \printbibliography

\end{document}